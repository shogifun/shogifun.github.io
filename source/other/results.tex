\newpage\section*{ЗАКЛЮЧЕНИЕ}\addcontentsline{toc}{section}{ЗАКЛЮЧЕНИЕ}
В реферате были изучены програмные средства и библиотеки, которые могут применяться для построения моделей машинного обучения, решающих задачу прогнозирования.Использование описанных алгоритмов и программных средств позволило построить модели, решающие задачу теннисного прогнозирования.Точность полученных моделей составила порядка 75 процентов.

Использование современных интегрированных сред разработки, таких как PyCharm, Spyder и Jypiter Notebook позволяют эффективно отбирать данные, строить алгоритмы на их основе, а также эффективно анализировать и визуализировать данные. Имользование системы контроля версий Git позволяет удобно следить за историй изменения програмного кода, а также коллаборироваться с другими исследователями. 

Система коипютерной верстки TeX, в частности ее набор макрорасширений Latex позволяют быстро создавать текстовые документы с вставкой в них  математических и химических формул, графиков, таблиц и других мультимедийных элементов.

