\newpage
\section*{ВВЕДЕНИЕ}
\addcontentsline{toc}{section}{ВВЕДЕНИЕ}
В последние несколько лет все больше разнообразных повседневных задач решаются с помощью алгоритмов машинного обучения. Машинное обучение (англ. Machine Learning, ML) — класс методов искусственного интеллекта, характерной чертой которых является не прямое решение задачи, а обучение в процессе применения решений множества сходных задач. Для построения таких методов используются средства математической статистики, численных методов, методов оптимизации, теории вероятностей, теории графов, различные техники работы с данными в цифровой форме. Теоретические основы машинного обучения были заложены еще в середине прошлого века, однако в течение долгого времени эта область компьютерной науки практически не развивалась, так для использования алгоритмов машинного обучения необходимы были высокопроизводительные вычислительные системы. С достижением необходимой производительности в начале 2010-х алгоритмы машинного обучения стали все чаще использоваться в решении разнообразных практических задач, таких как распознавание речи, жестов и образов; техническая и медицинская диагностика, обнаружение спама и другие.  
Одной из областей машинного обучения являются алгоритмы прогнозирования. Приведем несколько примеров решенных задач прогнозирования и алгоритмов, применяемых для этого:
\begin{enumerate}
	\item
Прогнозирование финансовых процессов и биржевых индексов. В частности, Дегтярев В.М использовал многослойные нейронные сети для прогнозирования поведения валютной пары доллар США/ швейцарский франк и в результате построил модель для торговли на бирже, прибыль работы с при помощи которой составляла порядка 7 процентов \cite{one}. Также можно выделить работу Samuel Edet, который использовал рекуррентные нейронные сети для прогнозирования изменений значения индеков SA SnP 500\cite{two}. Точность полученной им модели составляла порядка 75 процентов.	
	\item
Медицинская диагностика. Примером работ в этой области можно привести можно привести работу сотрудников университета Стэнфорда, которые использовали сверточную нейронную сеть для прогнозирования возможной аритмии у пациентов по данным кардиограммы.\cite{cardio}
	\item
Спортивное прогнозирование. Алгоритмы машинного обучения использовались для прогнозирования результатов матчей в самых разных видах спорта. В частности, Kahn использовал многослойную нейронную сеть для построения модели классификации для прогнозирования результатов матчей NFL в 2003 году. В качестве обучающей выборки были использованы первые 192 матча сезона. В результате результаты прогноза модели превзошли результаты предсказаний профессиональных экспертов. \cite{four}
\end{enumerate}
В качестве задачи прогнозирования была выбрана задача теннисного прогнозирования.
\section*{ОСНОВНЫЕ ПОНЯТИЯ И ОПРЕДЕЛЕНИЯ}
\addcontentsline{toc}{section}{Основые определения и понятия}
Машинное обучение (англ. machine learning, ML) — класс методов искусственного интеллекта, характерной чертой которых является не прямое решение задачи, а обучение в процессе применения решений множества сходных задач. Для построения таких методов используются средства математической статистики, численных методов, методов оптимизации, теории вероятностей, теории графов, различные техники работы с данными в цифровой форме.

Теннис — это игра с ракеткой, в которую могут играть как один на один, так и парами. Для простоты сфокусируемся на предсказании результатов одиночных матчей. Дадим определение задачи теннисного прогнозирования.Пусть имеется следующая информация о предстоящем теннисном матче: имена участников, тип покрытия и позиции игроков в рейтинге ATP; а также историческая информация об уже сыгранных матчах. Необходимо построить модель на основе исторических данных, которая будет наилучшим образом прогнозировать результаты предстоящих матчей. Выбор данной задачи был сделан по нескольким причинам. Во-первых, теннис является одним из самых популярных видов спорта, а рынок ставок на теннис является одним из крупнейших. Во-вторых, на данный момент большинство средств для решения данной задачи используют в своей основе статистические модели, а алгоритмы машинного обучения для этой задачи практически не используются.

Обучающая выборка - набор данных для построения, обучения и анализа алгоритма машинного обучения. В случае задачи теннисного прогнозирования данная выборка представлена ранее сыгранными теннисными матчами и информацией о них.

