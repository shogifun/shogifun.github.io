\newpage
\section{СРЕДСТВА ДЛЯ НАПИСАНИЯ КОДА И НАПИСАНИЯ РАБОТЫ}
\subsection{Средства для написания кода}
Cамыми популярными средами написания кода на языке программирования Python являются следующие IDE(англ. Integrated Development Environment - интегрировання среда разработки): IntelliJ IDEA PyCharm, Jupiter Notebook расширение для IPython и Spyder.

PyCharm — интегрированная среда разработки для языка программирования Python. Предоставляет средства для анализа кода, графический отладчик, инструмент для запуска юнит-тестов и поддерживает веб-разработку на Django. PyCharm разработана компанией JetBrains[5] на основе IntelliJ IDEA. PyCharm работает под операционными системами Windows, Mac OS X и Linux. PyCharm Professional Edition имеет несколько вариантов лицензий, которые отличаются функциональностью, стоимостью и условиями использования.PyCharm Professional Edition является бесплатным для образовательных учреждений и проектов с открытым исходным кодом.Существует также бесплатная версия Community Edition, обладающая усеченным набором возможностей. Распространяется под лицензией Apache 2.

Ключевые возможности PyCharm:
\begin{itemize}
	\item статический анализ кода, подсветка синтаксиса и ошибок.
	\item навигация по проекту и исходному коду: отображение файловой структуры проекта, быстрый переход между файлами, классами, методами и использованиями методов.
	\item рефакторинг: переименование, извлечение метода, введение переменной, введение константы, подъём и спуск метода и т. д.
	\item инструменты для веб-разработки с использованием фреймворка Django
	\item встроенный отладчик для Python
	\item встроенные инструменты для юнит-тестирования
	\item разработка с использованием Google App Engine
	\item поддержка систем контроля версий: общий пользовательский интерфейс для Mercurial, Git, Subversion, Perforce и CVS с поддержкой списков изменений и слияния.
\end{itemize}

Spyder (ранее Pydee) — свободная и кроссплатформенная интерактивная IDE для научных расчетов на языке Python, обеспечивающая простоту использования функциональных возможностей и легковесность программной части. Spyder является частью модуля spyderlib для Python, основанного на PyQt4, pyflakes, rope и Sphinx, предоставляющего мощные виджеты на PyQt4, такие как редактор кода, консоль Python (встраиваемая в приложения), графический редактор переменных (в том числе списков, словарей и массивов).

Ключевые возможности Spyder:
\begin{itemize}
	\item редактор с подсветкой синтаксиса Python, C/C++ и Fortran
	\item динамическая интроспекция кода (с помощью rope) — автодополнение, переход к определению объекта по клику мыши
	\item нахождение ошибок на лету (с использованием pyflakes)
	\item поддержка одновременного использования множества консолей Python (включая оболочку IPython)
	\item просмотр и редактирование переменных с помощью GUI (поддерживаются различные типы данных - числа, строки, списки, массивы, словари)
	\item встроенные средства доступа к документации (в формате Sphinx)
	\item гибко настраиваемый интерфейс
	\item интеграция с научными библиотеками Python - NumPy, SciPy, Matplotlib, Pandas.
\end{itemize}

IPython (англ. Interactive Python) — интерактивная оболочка для языка программирования Python, которая предоставляет расширенную интроспекцию, дополнительный командный синтаксис, подсветку кода и автоматическое дополнение. Является компонентом пакетов программ SciPy и Anaconda. IPython позволяет осуществлять неблокирующее (англ. non-blocking) взаимодействие с Tkinter, GTK, Qt и WX. Стандартная библиотека Python включает лишь Tkinter. IPython может интерактивно управлять параллельными кластерами, используя асинхронные статусы обратных вызовов и/или MPI. IPython может использоваться как замена стандартной командной оболочки операционной системы, особенно на платформе Windows, возможности оболочки которой ограничены. Поведение по умолчанию похоже на поведение оболочек UNIX-подобных систем, но тот факт, что работа происходит в окружении Python, позволяет добиваться большей настраиваемости и гибкости.
Начиная с версии 4.0, монолитный код был разбит на модули, и независимые от языка модули были выделены в отдельный проект Jupyter. Наиболее известной веб-оболочкой для IPython является Jupyter Notebook (ранее известный как IPython Notebook), позволяющая объединить код, текст и диаграммы, и распространять их для других пользователей.

Для хранения исходного кода была использована система контроля версий Git. Git — распределённая система управления версиями. Проект был создан Линусом Торвальдсом для управления разработкой ядра Linux, первая версия выпущена 7 апреля 2005 года. На сегодняшний день его поддерживает Джунио Хамано. Среди проектов, использующих Git — ядро Linux, Swift, Android, Drupal, Cairo, GNU Core Utilities, Mesa, Wine, Chromium, Compiz Fusion, FlightGear, jQuery, PHP, NASM, MediaWiki, DokuWiki, Qt, ряд дистрибутивов Linux.Программа является свободной и выпущена под лицензией GNU GPL версии 2. По умолчанию используется TCP порт 9418.

Система спроектирована как набор программ, специально разработанных с учётом их использования в сценариях. Это позволяет удобно создавать специализированные системы контроля версий на базе Git или пользовательские интерфейсы. Например, Cogito является именно таким примером оболочки к репозиториям Git, а StGit использует Git для управления коллекцией исправлений (патчей).Git поддерживает быстрое разделение и слияние версий, включает инструменты для визуализации и навигации по нелинейной истории разработки. Как и Darcs, BitKeeper, Mercurial, Bazaar и Monotone[en], Git предоставляет каждому разработчику локальную копию всей истории разработки, изменения копируются из одного репозитория в другой.Удалённый доступ к репозиториям Git обеспечивается git-демоном, SSH- или HTTP-сервером. TCP-сервис git-daemon входит в дистрибутив Git и является наряду с SSH наиболее распространённым и надёжным методом доступа. Метод доступа по HTTP, несмотря на ряд ограничений, очень популярен в контролируемых сетях, потому что позволяет использовать существующие конфигурации сетевых фильтров.

\subsection{Средства для написания реферата}
В качестве системы текстовой верстки был использован Latex. Latex — наиболее популярный набор макрорасширений (или макропакет) системы компьютерной вёрстки TeX, который облегчает набор сложных документов. В типографском наборе системы TeX форматируется традиционно как Latex.

Пакет позволяет автоматизировать многие задачи набора текста и подготовки статей, включая набор текста на нескольких языках, нумерацию разделов и формул, перекрёстные ссылки, размещение иллюстраций и таблиц на странице, ведение библиографии и др. Кроме базового набора существует множество пакетов расширения Latex. Первая версия была выпущена Лесли Лэмпортом в 1984 году; текущая версия, Latex2e, после создания в 1994 году испытывала некоторый период нестабильности, окончившийся к концу 1990-х годов, а в настоящее время стабилизировалась (хотя раз в год выходит новая версия).

Общий внешний вид документа в LaTeX определяется стилевым файлом. Существует несколько стандартных стилевых файлов для статей, книг, писем и т. д., кроме того, многие издательства и журналы предоставляют свои собственные стилевые файлы, что позволяет быстро оформить публикацию, соответствующую стандартам издания.

Возможности системы, не ограничены, благодаря механизму программирования новых макросов. Вот список некоторых возможностей, предлагаемых стандартными макросами и теми, которые можно скачать с сервера CTAN:
\begin{itemize}
	\item алгоритмы расстановки переносов, определения междусловных пробелов, балансировки текста в абзацах;
	\item автоматическая генерация содержания, списка иллюстраций, таблиц и т. д.;
	\item механизм работы с перекрёстными ссылками на формулы, таблицы, иллюстрации, их номер или страницу;
	\item механизм цитирования библиографических источников, работы с библиографическими картотеками;
	\item размещение иллюстраций (иллюстрации, таблицы и подписи к ним автоматически размещаются на странице и нумеруются);
	\item оформление математических формул, возможность набирать многострочные формулы, большой выбор математических символов;
	\item оформление химических формул и структурных схем молекул органической и неорганической химии;
	\item оформление графов, схем, диаграмм, синтаксических графов;
	\item оформление алгоритмов, исходных текстов программ (которые могут включаться в текст непосредственно из своих файлов) с синтаксической подсветкой;
	\item разбивка документа на отдельные части (тематические карты).
\end{itemize}
Расширенные средства работы с библиографическими данными предоставляются программой BibTeX. Базовые возможности работы с математическими формулами расширяются с помощью пакета AMS-LaTeX.

\subsection{Выводы}
Современные интегрированные среды разработки вкупе с системой контроля версий позволяют эффективно использовать средства языка Python  и дополнительных библиотек при разработки моделей алгоритмов для машинного обчуения.

Текствоый процессор Latex позволяет эффективно создавать и форматировать документы.