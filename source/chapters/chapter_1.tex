\newpage
\section{ИНСТРУМЕНТЫ ДЛЯ АНАЛИЗА И ОБРАБОТКИ ДАННЫХ}

\subsection{Общая схема решения задачи прогнозирования}
Приведем шаги для решения задачи прогнозирования:
\begin {enumerate}
	\item Поиск и фильтрация данных, построение обучающей выборки;
	\item Отбор наиболее релевантных признаков для построения обучающей выборки;	
	\item Построение и обучение модели;
	\item Анализ полученных результатов.
\end{enumerate}

В качестве языка программирования был использован Python. Python — высокоуровневый язык программирования общего назначения, ориентированный на повышение производительности разработчика и читаемости кода. Синтаксис ядра Python минималистичен. В то же время стандартная библиотека включает большой объём полезных функций. Python поддерживает несколько парадигм программирования, в том числе структурное, объектно-ориентированное, функциональное, императивное и аспектно-ориентированное. Основные архитектурные черты — динамическая типизация, автоматическое управление памятью, полная интроспекция, механизм обработки исключений, поддержка многопоточных вычислений и удобные высокоуровневые структуры данных. Код в Python организовывается в функции и классы, которые могут объединяться в модули (они в свою очередь могут быть объединены в пакеты).

Выбор в пользу данного языка был обусловлен наличием на нем большого количества библиотек, предназначенных для разработки и обучения алгоритмов машинного обучения. Важным преимуществом этих библиотек является наличие имплементаций низкоуровневых операций на языке С, что благоприятно влияет на производительность. 

Проведем обзор иструментов и библиотек, которые были использованы для решения поставленных выше задач.

\subsection{Построение обучающей выборки}
Для построения обучающей выборки из источника 1 была загружена информация об мужских одиночных теннисных матчах, сыгранных под эгидой ATP, за 2004-2018 годы. Для фильтрации данных и построения выборки использовался фреймворк Pandas.

Pandas – библиотека языка Python, предназначенная для обработки и анализа данных. Работа библиотеки построена поверх библиотеки NumPy. Библиотекой предоставляются разнообразные структуры и алгоритмы числовыми данными и временными рядами. Дадим краткое описание возможностей библиотеки:
\begin{itemize}
	\item
объект DataFrame, позволяющий манипулировать индексированными двумерными данными;
	\item
возможность обмена файлами разнообразных типов, а также между структурами в оперативной памяти;
	\item
наличие инструментов объединения данных, а также возможность обработки недостающей информации;
	\item
переформатирование информации, в том числе создание сводных таблиц данных;
	\item
наличие возможности получения среза данных по индексу, возможность получения выборки из больших объемов информации.\cite{five}
\end{itemize}

\subsection{Отбор признаков. Построение и обучение алгоритмов.}
После построения набора признаков для каждого из матча полученной обучающей выборки из него с помощью были выбраны наиболее релевантные признаки. Для этого реализация алгоритма RFE(recursive feature elimination) из библиотеки Scikit-learn.

Scikit-learn – библиотека для машинного обучения, написанная на языке Python.  В нее включены разнообразные алгоритмы классификации, регрессии и кластеризации, такие как случайный лес, метод опорных векторов, k-means, DBSCAN и другие. Разработка библиотеки началась Дэвидом Корнапеу в рамках проекта Google Summer of Code. Позже оригинальная кодовая база была полностью переписана другими разработчиками. Релиз библиотеки состоялся в феврале 2010 года после того, как в проект включились студенты французского государственного института исследований в информатике и автоматике INRIA. На данный момент данная библиотека завоевала популярность, и разработка обновлений продолжается. Основная часть кодовой базы написана на Python, часть алгоритмов написана на Cython для достижения большей производительности. Поддержка метода нелинейных опорных векторов реализована с помощью обертки на Cython вокруг библиотеки LIBSVM, а поддержка линейной регрессии и линейного метода опорных векторов с помощью обертки вокруг библиотеки LIBLINEAR. В данной работе эта библиотека использовалась для построения моделей логистической регрессии и метода опорных векторов, а также для отбора наиболее релевантных признаков с помощью алгоритма RFE.

После получения набора наиболее релевантных признаков с их помощью были построены 3 модели на основании следующих алгоритмов:
\begin{itemize}
\item
логистическая регрессия - это статистическая модель, используемая для предсказания вероятности возникновения некоторого события путём подгонки данных к логистической кривой?;
\item
искусственная нейронная сеть (ИНС) — математическая модель, а также её программное или аппаратное воплощение, построенная по принципу организации и функционирования биологических нейронных сетей — сетей нервных клеток живого организма. Это понятие возникло при изучении процессов, протекающих в мозге, и при попытке смоделировать эти процессы;
\item
метод опорных векторов (англ. SVM, support vector machine) — набор схожих алгоритмов обучения с учителем, использующихся для задач классификации и регрессионного анализа. Принадлежит семейству линейных классификаторов и может также рассматриваться как специальный случай регуляризации по Тихонову. Особым свойством метода опорных векторов является непрерывное уменьшение эмпирической ошибки классификации и увеличение зазора, поэтому метод также известен как метод классификатора с максимальным зазором.
\end{itemize}

Для построения указанных алгоритмов использовалась выше описанная библиотека Scikit-learn, а также библиотека Keras.

Keras – открытая библиотека на языке Python, предназначенная для построения нейросетевых моделей. Представляет собой интерфейс, надстроенный над Deeplearning4j, TensorFlow и Theano. Основным предназначением библиотеки является работа с нейронными сетями глубокого обучения. Библиотека было представлена в рамках проекта ONEIROS. Основным автором библиотеки является инженер Google Франсуа Шолле. Keras содержит в себе все основные реализации широко применяемых при создании нейронных сетей элементов, такие как слои, целевые и передаточные функции, оптимизаторы, а также набор инструментов для упрощения работы с картинками и текстом. Исходный код библиотеки размещен в открытом репозитории на Github. В работе данная модель использовалась для построения и обучения нейронной сети.

Для демострации работы моделей необходимо построить программу с графическим интерфейсом. Для его построения была использована библиотека Qt. Qt — кроссплатформенный фреймворк для разработки программного обеспечения на языке программирования C++. Есть также «привязки» ко многим другим языкам программирования: Python — PyQt, PySide; Ruby — QtRuby; Java — Qt Jambi; PHP — PHP-Qt и другие.

Со времени своего появления в 1996 году библиотека легла в основу многих программных проектов. Кроме того, Qt является фундаментом популярной рабочей среды KDE, входящей в состав многих дистрибутивов Linux.

Qt позволяет запускать написанное с его помощью программное обеспечение в большинстве современных операционных систем путём простой компиляции программы для каждой системы без изменения исходного кода. Включает в себя все основные классы, которые могут потребоваться при разработке прикладного программного обеспечения, начиная от элементов графического интерфейса и заканчивая классами для работы с сетью, базами данных и XML. Является полностью объектно-ориентированным, расширяемым и поддерживающим технику компонентного программирования.

Отличительная особенность — использование метаобъектного компилятора — предварительной системы обработки исходного кода. Расширение возможностей обеспечивается системой плагинов, которые возможно размещать непосредственно в панели визуального редактора. Также существует возможность расширения привычной функциональности виджетов, связанной с размещением их на экране, отображением, перерисовкой при изменении размеров окна.

\subsection{Выводы}
Средства языка Python, а также дополнительные библиотеки такие как Pandas, Scikit-lear, Keras и QT позволяют эффективно строить модели на основе алгоритмов машинного обучения для решения разнообразных задач прогнозировани, а также графический интерфейс для демонстрации работы полученных моделей.

